\chapter{背景}\label{chap:background}

第二章介绍了本工作所涉及到的一些工程技术方面的背景知识,包括Tilelink总线协议,芯片验证的基础知识以及业界常用的UVM验证方法学等。这部分背景知识补齐了一些技术细节,帮助读者更好地理解本工作的后续内容。

\section{TileLink一致性协议}

\subsection{TileLink简介}

TileLink是芯片级的互连标准,为多个主设备提供对内存和其他从设备的一致性的内存映射访问。TileLink作为快速可扩展互连协议,可提供低延迟和高吞吐量的传输。它被设计用于在片上系统(SoC)中连接通用多处理器,协处理器,加速器,DMA引擎以及简单或复杂的设备。
TileLink是类似ARM AXI和ACE的,通用的、支持一致性的总线协议以及互联标准。
TileLink的一些重要特性包括:
\begin{enumerate}
	\item 标准免费开放
	\item 为RISC-V设计,但也支持其他的ISA
	\item 为任意的缓存(如处理器核)或非缓存主设备(如DMA master)提供一致性的内存访问,支持类似MESI的一致性协议
\end{enumerate}

这些重要特性,让TileLink在开源芯片设计中广受欢迎。TileLink相关IP(如CrossBar,Arbiter,SerDes,NoC等)齐全并且很成熟,同时大量的开源芯片设计都使用TileLink总线作为访存接口,如Rocket Chip,Boom等。

为了支持各种不同设备从简单到复杂的访存需求,TileLink规范定义了三种协议兼容级别,这些级别说明了设备必须支持协议的哪个子集,如下表所示。
最简单的是TileLink轻量级不缓存(TL-UL),仅支持简单的内存读写(Get/Put)单个字的操作。它可以用于简单的,只需要单个字读写访问的,如UART之类外设。
接下来最复杂的是TileLink重量级不缓存(TL-UH),它添加了各种Hint(提示),原子操作和突发访问,但不支持一致性缓存。它可以用于稍微复杂一点的,需要较高数据吞吐量,同时又不需要自己硬件上维护一致性的主设备,如ICache,DMA master等。
最后,TileLink缓存(TL-C)是完整的协议,它支持使用一致性的缓存。它可以用于完整的,自己维护硬件一致性的cache,如DCache,L2,L3等各级Cache。

\begin{center}
\begin{tabular}{|l|l|l|l|}
\hline
                      & \textbf{TL-UL} & \textbf{TL-UH} & \textbf{TL-C} \\ \hline
Read/Write operations & $\checkmark$   & $\checkmark$   & $\checkmark$  \\ \hline
Multibeat messages    & $\checkmark$   & $\checkmark$   & $\checkmark$  \\ \hline
Atomic operations     & $\checkmark$   & $\checkmark$   & $\checkmark$  \\ \hline
Hint operations       & $\times$       & $\checkmark$   & $\checkmark$  \\ \hline
Cache block transfers & $\times$       & $\times$       & $\checkmark$  \\ \hline
Channels B+C+E        & $\times$       & $\times$       & $\checkmark$  \\ \hline
\end{tabular}
\end{table}

我感觉应该把channel,message,transaction合并成一页,然后再把coherence搞一页。
关于tilelink,大家主要要理解啥呢?tilelink有哪些流程,它们走的什么通道。每个通道是用来干啥的,就OK了吧?
其他太过于细节的,其实是不必要的啊。

channel上是message,message构成transaction,transaction支持请求的流程。然后对于这些流程,我们只需要着重介绍read/write,还有cache block transfer就好了吧。
这个应该怎么介绍了呢,自底向上或者自顶向下都是可以的吧。
随便介绍一下就OK了吧。

\subsection{TileLink通道、消息、事务和一致性}

上一节介绍了TileLink的主要特性,这一节主要介绍TileLink的一些技术细节,包括它的通道定义,消息,事务以及如何实现一致性。
这些内容是具有递进关系的。
通道是底层的逻辑链路,通道承载了消息。
消息往往是单向(从master到slave或者从slave到master)的信息流动。例如,读请求消息(从master到slave)或者读数据回复(从slave到master)。
事务是一个最基础的请求流程,例如读和写。多个逻辑上相关的消息一起组成了一个事务,例如,读由读请求和读数据回复组成;写由写请求和写回复组成。
而一致性则是系统处理所有的事务时满足的性质。要保证数据的一致性,需要许多事务的协同。
我们将按照自底向上的方式,从最底层的通道开始介绍,一直到最顶层的一致性相关的内容。


\subsubsection{TileLink通道}

TileLink协议总共有五条通道,分别是A、B、C、D和E。其中A、D通道是必须的,B、C和E通道是可选的,只有支持一致性协议的TL-C才需要,TL-UL和TL-UH并不需要。

两条用来进行访存操作的基础的通道是:

\begin{itemize}
	\item 通道A:发送一个请求,要求在指定的地址范围上执行操作,以访问或缓存数据。
	\item 通道D:向原始请求者发送数据响应或确认消息。
\end{itemize}

最高协议一致性级别(TL-C)添加了三个附加通道,这些通道提供了管理缓存数据块权限的功能:
\begin{itemize}
	\item 通道B:发送一个请求,要求在master缓存的地址上执行操作,以访问或写回该缓存的数据。
	\item 通道C:响应请求,发送数据或确认消息。
	\item 通道E:从原始请求者发送的缓存块传输的最终确认,用于序列化。
\end{itemize}

\subsubsection{TileLink消息}

由于TileLink支持的功能很多,支持的消息种类也很多,因此对于不同的通道,我们简要介绍几个消息类型。

\paragraph{通道A}
\begin{itemize}
	\item 
	\item 
\end{itemize}

\paragraph{通道D}
\begin{itemize}
	\item 
	\item 
\end{itemize}

\paragraph{通道B}
\begin{itemize}
	\item 
	\item 
\end{itemize}

\paragraph{通道C}
\begin{itemize}
	\item 
	\item 
\end{itemize}

\paragraph{通道E}
\begin{itemize}
	\item 
	\item 
\end{itemize}

\subsubsection{TileLink事务}

\subsubsection{TileLink一致性}

tilelink中定义的coherence状态。
我们选用的目录一致性协议。
可以给一下手册里面的几个例子。

这里在介绍一致性协议时可以区分update based和invalidation based,这两种都是可以的。
然后介绍清楚我们用的是哪一种啊。

\section{TileLink一致性协议}
### Verification
#### Verification的基础知识

首先验证是什么东西,验证是在干什么。

Verification的重要性。

验证的流程:specification(功能详述),验证计划,开发验证环境,测试,回归测试,芯片生产,硅后系统测试,逃逸分析。
本验证工作主要关注的是硅前的阶段,即在流片前协助调试,尽可能地发现bug。

验证的层次

Functional Verification是在做什么东西。
有哪些验证的方法,各有什么优缺点。

动态验证
静态验证

现在大家常用的验证方法。

#### UVM验证方法学

UVM是什么东西。是用来产生testbench的一种方法学。
UVM structure。




\subsection{现有cache设计的不足}
