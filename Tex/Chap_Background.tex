\chapter{背景}\label{chap:background}

第二章介绍了本工作所涉及到的一些工程技术方面的背景知识,包括Tilelink总线协议,芯片验证的基础知识以及业界常用的UVM验证方法学等。这部分背景知识补齐了一些技术细节,帮助读者更好地理解本工作的后续内容。

\section{TileLink一致性协议}

\subsection{TileLink简介}

TileLink是芯片级的互连标准,为多个主设备提供对内存和其他从设备的一致性的内存映射访问。TileLink作为快速可扩展互连协议,可提供低延迟和高吞吐量的传输。它被设计用于在片上系统(SoC)中连接通用多处理器,协处理器,加速器,DMA引擎以及简单或复杂的设备。
TileLink是类似ARM AXI和ACE的,通用的、支持一致性的总线协议以及互联标准。
TileLink的一些重要特性包括:
\begin{enumerate}
	\item 标准免费开放
	\item 为RISC-V设计,但也支持其他的ISA
	\item 为任意的缓存(如处理器核)或非缓存主设备(如DMA master)提供一致性的内存访问,支持类似MESI的一致性协议
\end{enumerate}

这些重要特性,让TileLink在开源芯片设计中广受欢迎。TileLink相关IP(如CrossBar,Arbiter,SerDes,NoC等)齐全并且很成熟,同时大量的开源芯片设计都使用TileLink总线作为访存接口,如Rocket Chip,Boom等。

为了支持各种不同设备从简单到复杂的访存需求,TileLink规范定义了三种协议兼容级别,这些级别说明了设备必须支持协议的哪个子集,如下表所示。
最简单的是TileLink轻量级不缓存(TL-UL),仅支持简单的内存读写(Get/Put)单个字的操作。它可以用于简单的,只需要单个字读写访问的,如UART之类外设。
接下来最复杂的是TileLink重量级不缓存(TL-UH),它添加了各种Hint(提示),原子操作和突发访问,但不支持一致性缓存。它可以用于稍微复杂一点的,需要较高数据吞吐量,同时又不需要自己硬件上维护一致性的主设备,如ICache,DMA master等。
最后,TileLink缓存(TL-C)是完整的协议,它支持使用一致性的缓存。它可以用于完整的,自己维护硬件一致性的cache,如DCache,L2,L3等各级Cache。

TL-UL和TL-UH中的消息,都是master请求slave执行一些读写预取操作,master没有获取数据块的权限,不要求将数据块的所有权从slave转移到master这边。因此它们常常用于master自己不需要缓存数据块(如DMA engine),或者master自己缓存的数据块是软件维护一致性,不需要硬件维护一致性(例如ICache)的情况。由于master并不请求数据块权限以缓存在本地,所以这些请求被称为uncached请求,这也是两个协议名称中U的由来。

TL-UL,TL-UH和TL-C三种兼容性级别,低级别是高级别的真子集,高级别包含了低级别的所有通道,所有消息(低级别的消息在高级别中功能可能会被扩展)。
兼容性级别越高,支持的消息越来越多,支持的功能越来越强。相应的,master和slave端的实现也就更加复杂。


\begin{center}
\begin{tabular}{|l|l|l|l|}
\hline
                      & \textbf{TL-UL} & \textbf{TL-UH} & \textbf{TL-C} \\ \hline
Read/Write operations & $\checkmark$   & $\checkmark$   & $\checkmark$  \\ \hline
Multibeat messages    & $\checkmark$   & $\checkmark$   & $\checkmark$  \\ \hline
Atomic operations     & $\checkmark$   & $\checkmark$   & $\checkmark$  \\ \hline
Hint operations       & $\times$       & $\checkmark$   & $\checkmark$  \\ \hline
Cache block transfers & $\times$       & $\times$       & $\checkmark$  \\ \hline
Channels B+C+E        & $\times$       & $\times$       & $\checkmark$  \\ \hline
\end{tabular}
\end{table}

我感觉应该把channel,message,transaction合并成一页,然后再把coherence搞一页。
关于tilelink,大家主要要理解啥呢?tilelink有哪些流程,它们走的什么通道。每个通道是用来干啥的,就OK了吧?
其他太过于细节的,其实是不必要的啊。

channel上是message,message构成transaction,transaction支持请求的流程。然后对于这些流程,我们只需要着重介绍read/write,还有cache block transfer就好了吧。
这个应该怎么介绍了呢,自底向上或者自顶向下都是可以的吧。
随便介绍一下就OK了吧。

\subsection{TileLink通道、消息、事务和一致性}

上一节介绍了TileLink的主要特性,这一节主要介绍TileLink的一些技术细节,包括它的通道定义,消息,事务以及如何实现一致性。
这些内容是具有递进关系的。
通道是底层的逻辑链路,通道承载了消息。
消息往往是单向(从master到slave或者从slave到master)的信息流动。例如,读请求消息(从master到slave)或者读数据回复(从slave到master)。
事务是一个最基础的请求流程,例如读和写。多个逻辑上相关的消息一起组成了一个事务,例如,读由读请求和读数据回复组成;写由写请求和写回复组成。
而一致性则是系统处理所有的事务时满足的性质。要保证数据的一致性,需要许多事务的协同。
我们将按照自底向上的方式,从最底层的通道开始介绍,一直到最顶层的一致性相关的内容。


\subsubsection{TileLink通道}

TileLink协议总共有五条通道,分别是A、B、C、D和E。其中A、D通道是必须的,B、C和E通道是可选的,只有支持一致性协议的TL-C才需要,TL-UL和TL-UH并不需要。

两条用来进行访存操作的基础的通道是:

\begin{itemize}
	\item 通道A:发送一个请求,要求在指定的地址范围上执行操作,以访问或缓存数据。
	\item 通道D:向原始请求者发送数据响应或确认消息。
\end{itemize}

最高协议一致性级别(TL-C)添加了三个附加通道,这些通道提供了管理缓存数据块权限的功能:
\begin{itemize}
	\item 通道B:发送一个请求,要求在master缓存的地址上执行操作,以访问或写回该缓存的数据。
	\item 通道C:响应请求,发送数据或确认消息。
	\item 通道E:从原始请求者发送的缓存块传输的最终确认,用于序列化。
\end{itemize}

\subsubsection{TileLink消息}

由于TileLink支持的功能很多,支持的消息种类也很多,因此对于不同的通道,我们简要介绍几种常用的消息类型。

\paragraph{通道A}
\begin{itemize}
	\item Get:Get消息是由master发出的请求,该请求希望访问特定的数据块以便读取它。
	\item PutFullData:PutFullData消息是由master提出的请求,该请求希望访问特定块的数据以便将其写入。
	\item ArithmeticData:ArithmeticData消息是由master提出的请求,该请求希望访问特定的数据块,并应用算术运算,对其进行读取-修改-写入的原子操作。
	\item Intent:Intent消息是由master发出的请求,它向slave传递了,它将来想要访问特定数据块的意图,这个可以用作预取请求。
    \item AcquireBlock:AcquireBlock消息是带有缓存的master所使用的消息类型,此消息用于获取一个数据块的副本,并在本地缓存。Master还可以使用此消息类型来升级其对已拥有的块的权限(例如获得对只读副本的写权限)。
\end{itemize}

\paragraph{通道D}
\begin{itemize}
	\item AccessAck:AccessAck充当原始请求代理的确认消息,这个确认消息是不带有数据的。
	\item AccessAckData:AccessAck充当原始请求代理的确认消息,这个确认消息是带有数据的。
	\item GrantData:GrantData消息是slave发出的,它既是一条请求信息,也是一条确认消息。它携带了确认信息以及数据块到原始请求master的数据副本。
	\item ReleaseAck:当slave收到来自master的Release[Data]消息时,slave发送ReleaseAck消息进行确认。
\end{itemize}

\paragraph{通道B}
\begin{itemize}
	\item ProbeBlock:ProbeBlock消息是slave用来查询或修改特定master缓存的数据块副本的权限的请求消息。一个可以slave撤销某个master对某块数据块的权限。
\end{itemize}

\paragraph{通道C}
\begin{itemize}
	\item ProbeAckData:当master收到来自slave的probe消息时,它使用ProbeAckData消息来发出确认,并写回脏数据。
	\item Release:Release消息是master用来主动降级其缓存的数据块的权限所使用的请求消息。
	\item ReleaseData:ReleaseData消息是master用来主动降级其缓存的数据块的权限,并写回脏数据所使用的请求消息。
\end{itemize}

\paragraph{通道E}
\begin{itemize}
	\item GrantAck:master使用GrantAck响应消息来提供对事务完成的最终确认。
\end{itemize}

\subsubsection{TileLink事务}

问题:这个图怎么画才好呢?
这个图太大了,得重画才好啊。
这个图要重画,可以拆分成几部分,或者干脆横过来啊。

本章我们介绍TileLink支持的事务,一个TileLink事务一般是一个独立的操作,这个操作涉及到master和slave之间发送和接收的多条消息。
一个事务是一个master和一个slave之间的操作。但是有时为了完成一个事务,可能后触发master和slave向其他节点发起其他事务。
例如,在一个层次化的cache系统中,master向slave发送get请求,发起一个读事务,下面的每一级cache都会向下一级cache发起一个读事务。=

下图展示了TileLink支持的所有操作,以及这些操作会涉及到哪些消息的发送与接收。
我们主要介绍五个常用操作,分别是普通的读写Get、Put以及用于支持一致性协议的,权限转移操作:Acquire、Probe、Release。

\begin{figure}[H] %H为当前位置,!htb为忽略美学标准,htbp为浮动图形
\centering %图片居中
\includegraphics[width=\textwidth]{Img/TileLink_Operations.PNG} %插入图片,[]中设置图片大小,{}中是图片文件名
\caption{Taxonomy of operations} %最终文档中希望显示的图片标题
\label{TileLink_Operations} %用于文内引用的标签
\end{figure}

\paragraph{Get操作}
Get操作是master对slave发起uncache的读,master只需要数据,而不需要缓存这个数据的权限。Get操作请求由master通过A通道发送Get消息,确认信息由slave通过D通道发送AccessAckData消息。

\paragraph{Put操作}
Put操作是master对slave发起uncache的写,master只需要把新数据写进slave就可以了,而不需要缓存这个数据的权限。Put操作请求由master通过A通道发送PutFullData或者PutPartialData消息,确认信息由slave通过D通道发送AccessAck消息。

\paragraph{Acquire操作}
Acquire操作是master对slave发起的申请块权限的操作。当操作完成后,master将获得块的特定权限,它将可以将数据块缓存在本地,并在本地执行权限允许的读写操作。
Acquire操作包含三步:
\begin{enumerate}
	\item master向slave发送Acquire消息,对于特定数据块申请特定的权限。
	\item slave向master回复Grant[Data]消息,对于此数据块,赋予master特定的权限(此权限可能比master请求的权限更高)。
	\item master使用GrantAck响应消息来提供对事务完成的最终确认。
\end{enumerate}

\paragraph{Probe操作}
Probe操作是slave对master发起的块权限查询或修改操作。当操作完成后,master将失去块的部分或全部权限。
Probe操作包含两步:首先slave向master发送ProbeBlock或ProbePerm消息,表明它希望将master缓存的特定数据块的降至特定的权限。接着,master回复ProbeAck[Data]来确认此消息,并写回可能的脏数据。

\paragraph{Release操作}
Release操作是master对slave发起的,主动的释放块权限和数据的操作。当操作完成后,master将失去块的部分或全部权限。
Release操作包含两步:首先master向slave发送Release[Data]消息,表明它将自己缓存的特定数据块降至特定的权限,同时写回可能有的脏数据。接着,slave回复ReleaseAck来确定此消息。

\section{TileLink与一致性协议}

在前面的小节中,我们介绍了TileLink中有哪些通道,哪些消息,哪些操作,其中有些就是和维护一致性有关的。在这一小节,我们完整介绍TileLink协议是如何维护一致性的。包括TileLink定义的块的权限,权限的转移以及操作定序等。
TileLink定义了缓存的数据块的权限包括None,Read以及Read + Write。
在TileLink中,对于一个数据块的所有缓存将构成了一棵树。树的根节点都位于缓存数据块的最终存储位置,例如DDR。根据节点在树中的不同位置,可以划分为四类:
\begin{itemize}
	\item Nothing:当前不缓存数据副本的节点,没有读取或写入权限。
	\item Trunk:在Tip和根节点之间的路径上具有缓存副本的节点。这类节点既没有读权限也没有写权限。例如,当写权限位于L1时,其下面的L2和L3就位于L1到根节点DDR的路径上,L2和L3就处于Trunk状态。
	\item Tip:缓存了数据副本节点,拥有最高权限,是负责对读写请求序列化的点。此状态下拥有对其副本的读/写权限,其中可能包含脏数据。
	\item Branch:在Tip上方的具有缓存副本的节点。对其副本具有只读权限。
\end{itemize}

下面我们可以简单对比一下TileLink协议的状态和MESI的状态,上述四种状态,无法严格对应到MESI协议,但大部分是可以对应过去的。
MESI协议中有四种状态,分别是:
\begin{itemize}
	\item Modified (M)
	\item Exclusive (E)
	\item Shared (S)
	\item Invalid (I)
\end{itemize}

Nothing和Invalid,Branch和Shared是可以一一对应的。Tip对应于Modified和Exclusive状态。TileLink在设计上,并不需要其他节点区分拥有写权限的节点有没有修改过数据,是不是dirty的。它们只需要只需要知道那个节点拥有最高权限就可以。至于是否修改过数据,只有那个节点自己需要关心,拥有最高权限的节点,可以自己内部设计有dirty bit,来记录自己有没有修改过数据,并根据这个bit来判断自己要不要写回脏数据。
Trunk状态是TileLink所独有的,这与TileLink的一致性协议设计是有关的。TileLink要求对于任何一块被缓存的数据块,所有缓存了它的节点必须组成一棵树,称为coherence tree。由于TileLink网络是有向无环图,同时被缓存的数据块唯一来自一个根节点(如DDR),需要组成树,实际上要求下级cache对上一级cache保持Inclusive的关系。这并不是说TileLink要求所有的cache都必须被设计成Inclusive Cache。下一级Cache并不需要真的缓存这个数据块,但是它需要知道自己处于Trunk状态,并且自己的子节点缓存了这个块。

在TileLink的网络中,节点之间通过执行权限转移操作,保证了数据的一致性以及对数据写的定序。
一般来说:
Acquire操作用来主动申请权限。
而Release操作用来主动地降低权限(例如当cache发生了替换,需要写时)。
Probe操作是一致性的invalidate请求,用来降低client的权限。

下面我们以一组例子说明在TileLink网络中,一致性是如何维护的:
\begin{figure}[H] %H为当前位置,!htb为忽略美学标准,htbp为浮动图形
\centering %图片居中
\includegraphics[width=\textwidth]{Img/TileLink_coherence.PNG} %插入图片,[]中设置图片大小,{}中是图片文件名
\caption{TileLink coherence examples} %最终文档中希望显示的图片标题
\label{TileLink_coherence} %用于文内引用的标签
\end{figure}

图中的树描述了一个系统中的多个缓存节点。最下方是DDR,最上方是L1 Cache。缓存了某个特定数据块的所有缓存节点构成了一棵树,以DDR为树根。
假设这是针对某个数据块d1的coherence tree。
图中的状态标识是:空白指Nothing状态,B指Branch状态,T指Trunk状态,TT是Tip状态。
L1 Cache从左到右标号分别为0,1,2,3。
L2 Cache从左到右标号分别为0,1,2,3。
L3 Cache从左到右标号分别为0,1。

图A到D,展示了系统中针对数据块d1的coherence tree的变化过程。

\begin{enumerate}
	\item 开始时,在图A的状态下,数据块d1存在于DDR中,系统中的其他cache都没有缓存d1,因此DDR节点是TT状态。
	\item L1 0请求了d1的写权限,因此从L1 0开始逐级向下通过Acquire操作升级权限,最终Tip状态从DDR转移到了L1 Cache 0,路径上的其他节点都变成了Trunk状态。Coherence Tree演化为了图B状态。
	\item L1 1通过Acquire申请d1的读权限,L2 1也向下转发请求,Acquire请求到达L3 0。L3 0发现自己处于Trunk状态,意味着它的某个孩子节点拥有Tip状态。因此L3 0向L2 0发送ProbeBlock请求,L2 0发现自己也是Trunk状态,因此将ProbeBlock向上发送到L1 0。L1 0收到ProbeBlock请求后,通过ProbeAckData消息将自己的权限降为Branch,并将脏数据写回到L2 0。L2 0也将ProbeAckData发送到L3 0,L2 0变为Branch状态。L3 0收到ProbeAckData后,回收了针对d1数据块的Tip权限,并通过GrantData,将数据和权限发送到L2 1,进而上传到L1 1,它们变为了Branch状态。Coherence Tree演化到了图C状态。
	\item L1 2通过Acquire申请d1的读权限,与前一步一样,最终Tip权限回到了DDR。值得注意的是,在图D中,L1 0和L2 0的Branch状态消失了,这不是L1 2的读导致的,可能是它们自己释放导致的。最终,Coherence Tree演化为了图D。
\end{enumerate}

### Verification
#### Verification的基础知识

首先验证是什么东西,验证是在干什么。

Verification的重要性。

验证的流程:specification(功能详述),验证计划,开发验证环境,测试,回归测试,芯片生产,硅后系统测试,逃逸分析。
本验证工作主要关注的是硅前的阶段,即在流片前协助调试,尽可能地发现bug。

验证的层次

Functional Verification是在做什么东西。
有哪些验证的方法,各有什么优缺点。

动态验证
静态验证

现在大家常用的验证方法。

#### UVM验证方法学

UVM是什么东西。是用来产生testbench的一种方法学。
UVM structure。




\subsection{现有cache设计的不足}
