\chapter{引言}\label{chap:introduction}


\section{研究背景}

随着诸如Dennard Scaling和摩尔定律等传统工艺技术演进所带来的收益的大大减少,计算机体系结构有望进入创新的黄金时代。领域专用架构(DSA)是一种有前途的解决方案,可以继续提高计算性能,同时能向以前工艺演进一样,带来能量和面积效率水平的极大提升。不幸的是,传统的芯片设计方法和硬件开发方法需要大量的非经常性工程成本(NRE成本),包括工具,人工,IP和时间等。这极大提升了芯片设计的门槛,最终阻碍了DSA的广泛采用。

而与之相对应的是,由于开源软件的激增,过去几十年软件开发的大量工程成本和极长的设计周期已大大缩减。一个小型的软件开发团队现在可以针对他们自己的用户群体的需求,使用更高级的工具,迅速实现他们创新的想法。受到软件社区的启发,开放式的硬件设计被认为是降低芯片设计成本的最有前途的方法之一,受到了学术界和工业界的普遍关注。对于学术界来说,与开源软件模拟器(例如gem5,MARSSx86,Sniper或ZSim)相比,开源硬件实现具有许多优势。与软件模型不同,硬件实现可以演示精确的微体系结构行为,运行包含大量指令数的实际应用程序,并凭提供功率和面积测量结果。此外,开放式硬件实现还提供了一个基准平台,作为新微体系结构优化的baseline。对工业界来说,开放的硬件标准意味着他们可以共建共享复杂的软件或硬件生态,而不必被标准专利构筑的门槛拦在市场之外。开放的硬件实现,则可以降低了他们的IP开发成本,缩短设计周期。

最近一些年,涌现出了大量开源芯片设计的工作,开源芯片设计的生态逐渐繁荣。处理器核,cache,外设,加速器,SoC等,都有了开源的设计实现。我们以Chipyard SoC集成框架为例,展示开源芯片生态的现状。Chipyard是用于对硬件进行全系统的设计和评估的一套框架。它包含了一系列工具、库以及开源IP,可以将现有的开源芯片设计以及商业的芯片设计整合成一个完整的SoC。
Chipyard包含有以下IP和模块:
\begin{itemize}
	\item 处理器核
		\begin{itemize}
			\item RocketCore:顺序单发射,支持RISC-V GC指令集,可运行Linux的处理器核
			\item BOOM core:乱序多发射,支持RISC-V GC指令集,可运行Linux的处理器核
		\end{itemize}
	\item 多核
		\begin{itemize}
			\item Rocketchip SoC框架
		\end{itemize}
	\item 加速器
		\begin{itemize}
			\item Hwacha:向量加速器
			\item Gemmini:神经网络加速器生成器
			\item NVLDA:神经网络推理加速器
			\item SHA3 RoCC Accelerator:SHA3哈希算法加速器
		\end{itemize}
	\item 设备
		\begin{itemize}
			\item IceNet:网卡(NIC)
			\item UART
			\item SPI Flash
		\end{itemize}
	\item 网络
		\begin{itemize}
			\item Tilelink SEREES: SerDes
			\item Tilelink Switch:Switch
			\item Tilelink Ring Network:NoC
		\end{itemize}
\end{itemize}



例如XXX(把PPT里那个SoC的图给放上来)。
从XXX到XXX。
但是没有一个高度可用的cache的实现。

\section{研究动机}

\subsection{现有cache设计的不足}


尽管最近确实有一些开源的cache设计,例如Sifive BlockInclusiveCache,XXX,TLSimpleL2,XXX。但是它们的性能,质量,通用性都不太好。这阻碍了开源芯片的广泛应用。
这里或许可以上一个表,列出现有的开源的cache在哪些方面有问题。

\subsection{cache设计的挑战}

在本工作中,我们试图解决这个问题,设计实现一个高性能,高质量,开源的cache生成器。设计一款cache生成器存在诸多挑战:

1. 高性能:
  * 延迟,吞吐(现代处理器大多是乱序,多核,这需要大量的MLP,各种预取,替换算法)
  * 延迟,频率
  * 命中率:各种预取,替换算法
2. 高质量:
  cache设计存在大量的复杂性,怎样把它搞对。复杂性体现在:
  * 一致性协议
  * 繁多的请求类型:cache请求,uncache请求,原子操作
  * 复杂的请求处理流程:(这个可以画一个L2的各种case,各种路径做个案例。最好画成树状的,就是我们有多少条不同的请求处理路径)
3. 通用性/灵活性
  现代处理中往往有多个层次的cache,L1,L2,L3,L4等。L1往往与核心是紧耦合的,而其他cache层次,大多与核心松耦合,它们往往有相同的接口:上下都是标准的总线协议,同时有相似的请求处理流程。因此作为一款cache生成器,我们希望它能尽可能地通用,同一份代码,不同的配置,就可以生成满足不同层次需求的cache。通用性带来的挑战主要是L2,L3,L4有不同的延、吞吐的性能需求,怎样用同一份代码满足不同的需求。

\section{论文的主要内容和贡献}

本工作设计实现了一款cache生成器:DeepDark,它是一款高性能,高质量,高灵活性的cache生成器,面向高性能处理器的缓存层次。GAIO以多种方法解决了Cache设计存在的多个挑战。

1. 高性能:
  * 常用路径与非常用路径分开,降低常用路径的延迟。
  * 多MSHR以提升吞吐
  * 与后端紧密结合的设计,确保频率达标
2. 高质量:
  * 基于UVM验证方法学的随机验证框架
  * 系统级测试
3. 高灵活性:
  * 高可配置性
  * 算法插件化,算法可配置

  我们使用chisel编程语言实现了DeepDark,它的延迟为XXX,吞吐达到XXX,在SMIC 14nm工艺下,频率高达2.0GHz。能够支持Linux,XXX等应用。

  易难的意见,贡献这里总结得不够好,可以先写其他的,最后再总结。

总结:本工作做出了三点微小的贡献:
1. 高性能cache的设计与实现
2. cache的验证
3. 高度灵活性,高度可配置,具备高通用性的cache结构设计

\section{论文组织}

本文的后续章节组织结构如下:

第二章介绍了本工作所涉及到的一些工程技术方面的背景知识,包括Tilelink总线协议,芯片验证的基础知识以及业界常用的UVM验证方法学等。这部分背景知识补齐了一些技术细节,帮助读者更好地理解本工作设计、实现以及验证部分的内容。

第三章介绍了DeepDark cache生成器的设计,包括架构设计,在此架构设计下请求的处理流程以及一些性能优化。

第四章介绍了DeepDark的实现。首先介绍了本工作基于的code base以及选用的语言。 接下来介绍了实现上的一些重要细节,例如主要模块的结构,它们的流水级、状态机;重要的同步、一致性协议、transaction control等问题在实现上是如何解决的等。最后还介绍了一些重要算法的实现。

第五章介绍了DeepDark cache生成器的验证。包括了cache需要验证的功能,cache验证的难点,cache验证方案的设计和cache验证环境的实现。

第六章介绍了DeepDark的评估。首先介绍了评估实验的配置以及实验的方法论。然后对cache的性能,后端实现,验证等各方面进行了综合的评估,并给出了结果。

第六章对相关工作进行总结,并展望了下一步的研究方向。
